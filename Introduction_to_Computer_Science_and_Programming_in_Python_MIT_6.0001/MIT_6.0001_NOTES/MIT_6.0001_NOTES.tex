			%++++++++++++++++++++++++++++++++++++++++
% Don't modify this section unless you know what you're doing!
\documentclass[letterpaper,12pt]{article}
\usepackage{tabularx} % extra features for tabular environment
\usepackage{amsmath}  % improve math presentation
\usepackage{graphicx} % takes care of graphic including machinery
\usepackage[margin=1in,letterpaper]{geometry} % decreases margins
\usepackage{cite} % takes care of citations
\usepackage[final]{hyperref} % adds hyper links inside the generated pdf file
\hypersetup{
	colorlinks=true,       % false: boxed links; true: colored links
	linkcolor=blue,        % color of internal links
	citecolor=blue,        % color of links to bibliography
	filecolor=magenta,     % color of file links
	urlcolor=blue         
}
%++++++++++++++++++++++++++++++++++++++++


\begin{document}

\title{Introduction to Computer Science and Programming in Python (MIT)}
\author{N. Ceylan}
\date{\today}
\maketitle



\section{Lecture 1: What is Computation?}

% C:\Users\eiree\AppData\Local\Programs\Python\Python37-32\python.exe .\2.1.py



\begin{description}

	\item[$\bullet$ int] İntegers
	\item[$\bullet$ float] Real numbers
		\item[$\bullet$ bool] Boolean
	\item[$\bullet$ NoneType] Special none value
	\item[$\bullet$ type()] can use type() to see type of an object.
	
	
\end{description}

\section{ Branching and Iteration}




\section{String Manipulation, Guess and Check, Approximations, Bisection}


\begin{description}
	
	\item[$\bullet$ $>>>$] s = "abc"
	\item[$\bullet$ $>>>$] len(s)
    \item[$\bullet$ $>>>$] 3


	
	
\end{description}

\section{Decomposition, Abstraction, and Functions}


in programming, divide code into modules:

\begin{description}
	
	\item[$\bullet$ ] are self-contained
	\item[$\bullet$ ] used to break up code
	\item[$\bullet$ ] intended to be reusable
		\item[$\bullet$ ] keep code organized
		\item[$\bullet$ ] keep code coherent
		
	
\end{description}


\section{Tuples, Lists, Aliasing, Mutability, and Cloning}


\subsection{Tuples}


\begin{description}
	
	\item[$\bullet$ ] Immutable
	\item[$\bullet$ $>>>$ ] tuple = ()
	
	
\end{description}







\subsection{List}

hot is an alias for warm - changing one changes another. \\
append() has a side effect.



\begin{description}
	
	\item[$\bullet$ $>>>$ ] a = 1
	\item[$\bullet$ $>>>$ ] b = a
\item[$\bullet$ $>>>$ ] print(a)
\item[$\bullet$ $>>>$ ] print(b)
\item[$\bullet$ $>>>$ ] 
\item[$\bullet$ $>>>$ ] warm = ["red", "yellow", "orange"]
\item[$\bullet$ $>>>$ ] hot = warm
\item[$\bullet$ $>>>$ ] hot.append("pink")
\item[$\bullet$ $>>>$ ] print(hot)
\item[$\bullet$ $>>>$ ] print(warm)
\item[$\bullet$ $>>>$ ]	["red", "yellow", "orange", "pink"]
\item[$\bullet$ $>>>$ ]	["red", "yellow", "orange", "pink"]
\end{description}



\section{Recursion and Dictionaries}



\section{Testing, Debugging, Exceptions, and Assertions}




\section{Object Oriented Programming}



\section{Python Classes and Inheritance}



\section{Understanding Program Efficiency, Part 1}

\section{Understanding Program Efficiency, Part 2}

\end{document}
